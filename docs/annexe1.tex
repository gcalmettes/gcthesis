\chapter{Coeur Isol� Perfus�, Notes Historiques}
\label{chap:histocoeurperf}

\malettrine{L}{e} concept de coeur isol� perfus� a �t� introduit pour la
  premi�re fois il y a plus
  de 150 ans par deux physiologistes, Carl Ludwig (1816 - 1895) et Elias Cyon
  (1842 - 1912) \citep{Zimmer:1998kx}.
  Ces pionniers mirent au point le premier syst�me de coeur isol� 
  perfus� sur la grenouille pour �tudier l'influence de la temp�rature sur la 
  fonction cardiaque \citep{Cyon:1866mc}.
  La perfusion des coeurs de mammif�res (dont l'alimentation du myocarde est
  assur� par le r�seau coronaire) ne 
  fut mise au point qu'une vingtaine d'ann�es plus tard sous l'influence de Henry
  Newell Martin (1848 - 1893) et surtout Oscar Langendorff (1853 - 1908) qui d�crivit en
  d�tail la technique et donna son nom � cette pr�paration exp�rimentale
  \citep{Langendorff:1895rm}.
  
  Le mod�le du coeur isol� perfus� s'est aujourd'hui impos� comme un outil de choix
  pour �tudier la
  fonction cardiaque et mieux comprendre les m�canismes qui la r�gissent.

